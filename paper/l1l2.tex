\documentclass[11pt]{article}

\linespread{1.09}
\sloppy
\sloppypar
\raggedbottom
\frenchspacing

\begin{document}

\section*{Map-making by convex optimization}

\noindent
DWH, JCH, others

\paragraph{abstract}
% context
How to combine multi-wavelength maps of the microwave sky to generate
maps of the cosmic microwave background primary anisotropies, the
thermal and kinetic Sunyaev--Zeldovich effects, point sources, and
Milky Way foregrounds?
Each of these has different dependences on wavelength, and different
spatial structure.
But because we have only one sky, they can't be separated cleanly at
high confidence.
% aims
Nonetheless, for many purposes, we want to be able to make point
estimates of these maps from our very good all-sky data.
Here we consider map-making algorithms based on convex optimization.
% methods
We construct a family of objective functions based on L1 and L2 norms
of maps and residuals such that optimization is convex and the optimal
maps are maps of cosmological quantities of interest.
Being convex, optimization is reliable and straightforward.
% results
We find YYY.

\section{Method}

The assumptions of the method we will develop and test in this project
are as follows:
\begin{itemize}\itemsep=0ex
\item list
\item assumptions
\item here
\end{itemize}

The model is
\begin{eqnarray}
  y_{np} &=& \sum_{p'} B_{npp'}\,\left[
      \sum_{\ell m} R_{p'\ell m}\,t_{\ell m}
    + \sum_{k} A_{nk}\,x_{kp'}\right] + \sigma_{np}\,\epsilon_{np} \quad ,
\end{eqnarray}
where $y_{np}$ is the datum at wavelength $n$ in data pixel $p$,
$B_{npp'}$ is the beam-adjust\-ment matrix for frequency $n$, relating the response of data
pixel $p$ to model-plane (map-plane) pixel $p'$, $R_{p'\ell m}$ is the
spherical-harmonic transform from the $\ell m$ basis to the $p'$
basis, $t_{\ell m}$ is the temperature map in the $\ell m$ basis,
$A_{nk}$ is the design matrix containing every possible foreground or
secondary anisotropy's relative contribution at wavelength $n$, the
$x_{kp'}$ are the secondary and foreground maps, the $\sigma_{np}$ are
the individual-pixel noise root-variances, and the $\epsilon_{np}$ are
the normalized residuals (or pixel-noise contributions).

One key idea is that the $A_{nk}$ design matrix will not just include the
spectral dependence of thermal S--Z, dust, and synchrotron and so
on. The idea is that it will contain every sensible frequency
dependence of any possible sensible component. That's what supports
the sparsity of the $x_{kp'}$ maps: Even if there is dust emission
over most of the sky, it will be dust at different temperatures in
different places.

Another key idea is that the maps $x_{kp'}$ will be reconstructed on a
pixel (or other!) basis that is higher (or no lower) resolution than
the highest resolution data. This is a deconvolution model, in some
sense.  The beam-adjust\-ment matrices $B_{npp'}$ are kernels that take
the maps on the high resolution $p'$ basis down to the lower
resolution $p$ basis with the real point-spread function (or beam) at
frequency $n$. For information-theoretic reasons, it would be a bad
idea to make the $p'$ basis far higher resolution than the $p$ basis
unless the number of wavelength maps $n$ is very large and the beams
are extremely well known.

While this method is very general, it does assume that the
beam-adjust\-ment matrices, transforms, and design matrix are perfectly
known. That is, it assumes that there aren't components that have
spectra unlike anything we have put into the design matrix (though we
can throw crazy shit in there), and there aren't beam issues we
haven't tracked and accounted for. For now, the method assumes that
all the frequencies $n$ are measured on the same pixel map $p$, but
that isn't a strong assumption; that is, it would be easy to relax
this assumption without much change to the method or code.

The objective function is
\begin{eqnarray}
  Q &=&
    \sum_{\ell m} C_{\ell m}^{-1}\,|t_{\ell m}|^2
  + \sum_{k} \Lambda_k\,\sum_{p'} |x_{kp'}|
  + \sum_{np} |\epsilon_{np}|^2 \quad ,
\end{eqnarray}
where the first term penalizes the temperature maps according to the
expected variances $C_{\ell m}$ from the cosmological model, the
second term asks each of the $K$ foreground and secondary maps $k$ to
be sparse, and the third term is (effectively) the likelihood
function.
This objective function is convex, so a gradient descent in the space
of the $t_{\ell m}$, the $x_{kp'}$, and the $\epsilon_{np}$ will
obtain a unique set of maps.
All the art will be in setting the L1 amplitudes $\Lambda_k$.

\end{document}
